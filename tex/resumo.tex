\begin{resumo}
Com a ascensão dos métodos ágeis de desenvolvimento nos últimos anos, diferentes técnicas de teste de software estão emergindo para dar suporte à principal característica de tais métodos: \textit{feedback} rápido. Por serem técnicas emergentes do meio empresarial e com sua utilização tendo início e crescimento nos últimos anos, ainda não existem muitas referências acadêmicas tratando do assunto. Este trabalho enfocará as técnicas Desenvolvimento Guiado por Testes (do inglês, \textit{Test-Driven Development}), Desenvolvimento Guiado por Comportamento (do inglês, \textit{Behaviour-Driven Development}), Integração Contínua e Dublês de Teste, abordando-as em um estudo de caso, agregando conhecimentos, ainda dispersos e difusos, sobre as diferentes abordagens, possibilidades e pontos em aberto no emprego de tais técnicas, utilizando para isso o projeto kanban-roots.
\end{resumo}