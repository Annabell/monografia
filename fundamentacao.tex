\chapter{Fundamentação teórica}

\section{Agilismo}
\label{sec:agilismo}

Em 2001 um grupo de dezessete especialistas, reconhecidos pela comunidade como grandes nomes do desenvolvimento software, se reuniram para discutir sobre um crescente conjunto de métodos que vinham surgindo e decidiram usar o termo Agilismo para descrever essa nova geração de métodos ágeis \cite{AgileStory}. Na mesma reunião, eles também escreveram o Manifesto Ágil \cite{AgileManifesto}, delineando um conjunto de valores e princípios que, em resumo, trilham um caminho para a eliminação de documentação e processos desnecessários, buscando a simplicidade, com foco na geração de valor e proximidade com o cliente, além de possibilitar respostas rápidas e eficazes às mudanças.

Pode-se dizer então, que o Desenvolvendo Ágil, ou Agilismo, é um rótulo genérico para os métodos de desenvolvimento de software baseados no Manifesto Ágil \cite{BDDRodrigo}.

\section{Testes automatizados}
\label{sec:testes_automatizados}
